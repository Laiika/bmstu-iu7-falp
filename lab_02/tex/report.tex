

\documentclass[12pt]{report}
\usepackage[utf8]{inputenc}
\usepackage[russian]{babel}
\usepackage[14pt]{extsizes}
\usepackage{listings}
\usepackage{graphicx}
\usepackage{amsmath,amsfonts,amssymb,amsthm,mathtools} 
\usepackage{pgfplots}
\usepackage{filecontents}
\usepackage{float}
\usepackage{indentfirst}
\usepackage{eucal}
\usepackage{enumitem}
%s\documentclass[openany]{book}
\frenchspacing

\usepackage{titlesec}
\titleformat{\section}
{\normalsize\bfseries}
{\thesection}
{1em}{}
\titlespacing*{\chapter}{0pt}{-30pt}{8pt}
\titlespacing*{\section}{\parindent}{*4}{*4}
\titlespacing*{\subsection}{\parindent}{*4}{*4}

\usepackage{indentfirst} % Красная строка

\usetikzlibrary{datavisualization}
\usetikzlibrary{datavisualization.formats.functions}

\usepackage{amsmath}

\usepackage{graphicx}
\newcommand{\img}[3] {
    \begin{figure}[h]
        \center{\includegraphics[height=#1]{img/#2}}
        \caption{#3}
        \label{img:#2}
    \end{figure}
}


% Для листинга кода:
\lstset{ %
	language=c,                 % выбор языка для подсветки (здесь это С)
	basicstyle=\small\sffamily, % размер и начертание шрифта для подсветки кода
	numbers=left,               % где поставить нумерацию строк (слева\справа)
	numberstyle=\tiny,           % размер шрифта для номеров строк
	stepnumber=1,                   % размер шага между двумя номерами строк
	numbersep=5pt,                % как далеко отстоят номера строк от подсвечиваемого кода
	showspaces=false,            % показывать или нет пробелы специальными отступами
	showstringspaces=false,      % показывать или нет пробелы в строках
	showtabs=false,             % показывать или нет табуляцию в строках
	frame=single,              % рисовать рамку вокруг кода
	tabsize=2,                 % размер табуляции по умолчанию равен 2 пробелам
	captionpos=t,              % позиция заголовка вверху [t] или внизу [b] 
	breaklines=true,           % автоматически переносить строки (да\нет)
	breakatwhitespace=false, % переносить строки только если есть пробел
	escapeinside={\#*}{*)}   % если нужно добавить комментарии в коде
}


\usepackage[left=3cm,right=1.5cm, top=2cm,bottom=2cm,bindingoffset=0cm]{geometry}
% Для измененных титулов глав:
\usepackage{titlesec, blindtext, color} % подключаем нужные пакеты
\definecolor{gray75}{gray}{0.75} % определяем цвет
\newcommand{\hsp}{\hspace{20pt}} % длина линии в 20pt
% titleformat определяет стиль
\titleformat{\chapter}{\LARGE\bfseries}{\thechapter}{20pt}{\LARGE\bfseries}
\titleformat{\section}{\Large\bfseries}{\thesection}{18pt}{\Large\bfseries}


% plot
\usepackage{pgfplots}
\usepackage{filecontents}
\usetikzlibrary{datavisualization}
\usetikzlibrary{datavisualization.formats.functions}

\begin{document}
	%\def\chaptername{} % убирает "Глава"
	\thispagestyle{empty}
	\begin{titlepage}
		\noindent \begin{minipage}{0.15\textwidth}
			\includegraphics[width=\linewidth]{img/b_logo}
		\end{minipage}
		\noindent\begin{minipage}{0.9\textwidth}\centering
			\textbf{Министерство науки и высшего образования Российской Федерации}\\
			\textbf{Федеральное государственное бюджетное образовательное учреждение высшего образования}\\
			\textbf{~~~«Московский государственный технический университет имени Н.Э.~Баумана}\\
			\textbf{(национальный исследовательский университет)»}\\
			\textbf{(МГТУ им. Н.Э.~Баумана)}
		\end{minipage}
		
		\noindent\rule{18cm}{3pt}
		\newline\newline
		\noindent ФАКУЛЬТЕТ $\underline{\text{«Информатика и системы управления»}}$ \newline\newline
		\noindent КАФЕДРА $\underline{\text{«Программное обеспечение ЭВМ и информационные технологии»}}$\newline\newline\newline\newline\newline
		
		\begin{center}
			\noindent\begin{minipage}{1.1\textwidth}\centering
				\Large\textbf{  Отчет по лабораторной работе №2}\newline
				\textbf{по дисциплине <<Функциональное и логическое}\newline
				\textbf{~~~программирование>>}\newline\newline
			\end{minipage}
		\end{center}
		
		\noindent\textbf{Тема} $\underline{\text{Определение функций пользователя~~~~~~~~~~~~~~~~~~~~~~~~~~}}$\newline\newline
		\noindent\textbf{Студент} $\underline{\text{Завойских Е.В.~~~~~~~~~~~~~~~~~~~~~~~~~~~~~~~~~~~~~~~~~~~~~~~~~~}}$\newline\newline
		\noindent\textbf{Группа} $\underline{\text{ИУ7-63Б~~~~~~~~~~~~~~~~~~~~~~~~~~~~~~~~~~~~~~~~~~~~~~~~~~~~~~~~~~~~}}$\newline\newline
		\noindent\textbf{Оценка (баллы)} $\underline{\text{~~~~~~~~~~~~~~~~~~~~~~~~~~~~~~~~~~~~~~~~~~~~~~~~~~~~~~~~~~~}}$\newline\newline
		\noindent\textbf{Преподаватели} $\underline{\text{Толпинская Н.Б., Строганов Ю.В.~~~~~~~~~~~~~}}$\newline\newline\newline
		
		\begin{center}
			\vfill
			Москва~---~\the\year
			~г.
		\end{center}
	\end{titlepage}
	
\chapter{Теоретические вопросы}
	
\section{Базис Lisp}

Базис --- это минимальный набор конструкций языка, на основе которого могут быть построены остальные.

Базис Lisp:

\begin{itemize}
	\item атомы;
        \item структуры;
	\item базовые функции и базовые функционалы.
\end{itemize}


\section{Классификация функций}

\begin{itemize}
	\item базисные функции;
        \item функции ядра;
	\item пользовательские функции.
\end{itemize} 

Классификация функций по аргументам и поведению:

\begin{itemize}
	\item чистые функции (фиксированное количество аргументов, для определенного набора аргументов один фиксированный результат);
        \item формы (переменное количество аргументов или аргументы обрабатываются по-разному);
	\item функционалы (принимают функцию в качестве аргумента или возвращают функцию).
\end{itemize} 

Классификация функций по назначению:

\begin{itemize}
	\item селекторы (car, cdr);
        \item конструкторы (cons, list);
	\item предикаты (atom, numberp);
        \item функции сравнения (eq, eql, equal).
\end{itemize} 


\section{Способы создания функций}

\begin{itemize}
    \item с использованием $\lambda$-нотации (функции без имени)
	
            $\lambda$-выражение: (lambda $\lambda$-список тело\_функции), 
            где $\lambda$-список --- формальные параметры функции.
	
            Вызов такой функции осуществляется следующим способом: ($\lambda$-выражение фактические параметры).

            Вычисление функций без имени может быть выполнено с использованием функционала apply: (apply $\lambda$-выражение список\_фактических\_параметров); или с использованием функционала funcall: (funcall $\lambda$-выражение фактические\_параметры).
	
    \item с использованием макро-определения defun: 
	
	(defun имя\_функции $\lambda$-выражение), 
	
	или  в облегченной форме:
	
	(defun имя\_функции $(x_1, x_2, ..., x_k)$ тело\_функции), 
	где $(x_1, x_2, ..., x_k)$ --- список аргументов.
	
	В качестве имени функции выступает символьный атом. 
	Вызов именованной функции осуществляется следующим образом: (имя\_функции фактические параметры).
\end{itemize} 


\section{Функции car и cdr, eq, eql, equal, equalp}

Функции car и cdr принимают точечную пару или список в качестве аргумента. Функция car возвращает голову (значение по первому указателю списковой ячейки). Функция cdr возвращает хвост (значение по второму указателю списковой ячейки).  

eq, eql, equal, equalp --- функции сравнения:

\begin{itemize}
	\item eq сравнивает два символьных атома. Возвращает T, когда значением одного из аргументов является атом, и одновременно значения аргументов равны. В ином случае возвращает Nil;
	\item eql сравнивает символьные атомы и числа одного типа. Например, (eql 3 3) -> T, (eql 3 3.0) -> Nil;
	\item equal работает идентично eql + сравнивает списки (считая списки эквивалентными, если они рекурсивно, согласно тому же equal, имеют одинаковую структуру и содержимое);
	\item equalp сравнивает символьные атомы, числа разных типов (и (equalp 1 1), и (equalp 1 1.0) вернет T) и списки. 
\end{itemize}


\section{Назначение и отличие в работе cons и list}

cons принимает 2 аргумента, создает списковую ячейку и ставит указатели на 2 аргумента, таким образом объединяя их в точечную пару.

Функция list, составляющая список из значений своих аргументов, создает столько списковых ячеек, сколько аргументов ей было передано. Эта функция относится к особым, поскольку у неё может быть произвольное число аргументов.

Основные отличия:

\begin{itemize}
	\item cons принимает фиксированное количество аргументов, list --- произвольное. 
	\item cons создает одну списковую ячейку, list --- список.
\end{itemize}

\chapter{Практические задания}	

\section{Составить диаграмму вычисления следующих выражений:}

\begin{enumerate}
    \item (equal 3 (abs -3));
    \item (equal (+ 1 2) 3);
    \item (equal (* 4 7) 21);
    \item (equal (* 2 3) (+ 7 2));
    \item (equal (- 7 3) (* 3 2));
    \item (equal (abs (- 2 4)) 3).
\end{enumerate}

Решение приложено к отчету на отдельном листе.


\section{Написать функцию, вычисляющую гипотенузу прямоугольного треугольника по заданным катетам и составить диаграмму ее вычисления}

\begin{lstlisting}[language=Lisp]
(defun hyp (a b) (sqrt (+ (* a a) (* b b))))
(hyp 3 4) -> 5
\end{lstlisting}

Диаграмма приложена к отчету на отдельном листе.
 

\section{Каковы результаты вычисления следующих выражений? (объяснить возможную ошибку и варианты ее устранения)}

\begin{enumerate}
    \item (list 'a c)
    
    ошибка: variable C has no value
        
    решение: добавление ' перед c

    \item (cons 'a (b c))

    ошибка: undefined function B
    
    решение: добавление ' перед (b c)

    \item (cons 'a '(b c))

    (A B C)

    \item (caddr (1 2 3 4 5))

    ошибка: 1 is not a function name; try using a symbol instead

    решение: добавление ' перед (1 2 3 4 5)

    \item (cons 'a 'b 'c)

    ошибка: too many arguments given to CONS: (CONS 'A 'B 'C)

    решение: замена 'b 'c на '(b c)

    \item (list 'a (b c))

    ошибка: undefined function B

    решение: добавление ' перед (b c)

    \item (list a '(b c))

    ошибка: variable A has no value

    решение: добавление ' перед a

    \item (list (+ 1 '(length '(1 2 3))))

    ошибка: (LENGTH '(1 2 3)) is not a number

    решение: убрать ' перед (length '(1 2 3))
\end{enumerate}


\section{Написать функцию $longer\_then$ от двух списков-аргументов, которая возвращает T, если первый аргумент имеет большую длину} 

\begin{lstlisting}[language=Lisp]
(defun longer_then (l1 l2) (> (length l1) (length l2)))
\end{lstlisting}


\section{Каковы результаты вычисления следующих выражений?}

\begin{enumerate}
    \item (cons 3 (list 5 6)) -> (3 5 6)

    \item (cons 3 '(list 5 6)) -> (3 list 5 6)

    \item (list 3 'from 9 'lives (- 9 3)) -> (3 from 9 lives 6)

    \item (+ (length for 2 too) (car '(21 22 23))) -> variable FOR has no value

    \item (cdr '(cons is short for ans)) -> (is short for ans)

    \item (car (list one two)) -> variable ONE has no value

    \item (car (list 'one 'two)) -> one

\end{enumerate}


\section{Дана функция (defun mystery (x) (list (second x) (first x))). Какие результаты вычисления следующих выражений?}

\begin{enumerate}
    \item (mystery (one two)) -> undefined function ONE

    \item (mystery (last one two)) -> variable ONE has no value

    \item (mystery free) -> variable FREE has no value

    \item (mystery one 'two) -> variable ONE has no value

\end{enumerate}


\section{Написать функцию, которая переводит температуру в системе Фаренгейта в температуру по Цельсию (defun f\_to\_c (temp)...)}

\begin{lstlisting}[language=Lisp]
(defun f-to-c (temp) (* (/ 5 9) (- temp 32.0)))
(f-to-c 451) -> 232.77779 
\end{lstlisting}


\section{Что получится при вычислении каждого из выражений?}

\begin{enumerate}
    \item (list 'cons t NIL) -> (cons t NIL)

    \item (eval (list 'cons t NIL)) -> (t)

    \item (eval (eval (list 'cons t NIL))) -> undefined function T

    \item (apply #cons ''(t NIL)) -> bad syntax for complex number: #CONS

    \item (eval NIL) -> NIL

    \item (list 'eval NIL) -> (eval NIL)

    \item (eval (list 'eval NIL)) -> NIL
    

\end{enumerate}

\bibliographystyle{utf8gost705u}  % стилевой файл для оформления по ГОСТу
\bibliography{51-biblio}          % имя библиографической базы (bib-файла)
	
\end{document}
