

\documentclass[12pt]{report}
\usepackage[utf8]{inputenc}
\usepackage[russian]{babel}
\usepackage[14pt]{extsizes}
\usepackage{listings}
\usepackage{graphicx}
\usepackage{amsmath,amsfonts,amssymb,amsthm,mathtools} 
\usepackage{pgfplots}
\usepackage{filecontents}
\usepackage{float}
\usepackage{indentfirst}
\usepackage{eucal}
\usepackage{enumitem}
%s\documentclass[openany]{book}
\frenchspacing

\usepackage{titlesec}
\titleformat{\section}
{\normalsize\bfseries}
{\thesection}
{1em}{}
\titlespacing*{\chapter}{0pt}{-30pt}{8pt}
\titlespacing*{\section}{\parindent}{*4}{*4}
\titlespacing*{\subsection}{\parindent}{*4}{*4}

\usepackage{indentfirst} % Красная строка

\usetikzlibrary{datavisualization}
\usetikzlibrary{datavisualization.formats.functions}

\usepackage{amsmath}

\usepackage{graphicx}
\newcommand{\img}[3] {
    \begin{figure}[h]
        \center{\includegraphics[height=#1]{img/#2}}
        \caption{#3}
        \label{img:#2}
    \end{figure}
}


% Для листинга кода:
\lstset{ %
	language=prolog,                 % выбор языка для подсветки 
	basicstyle=\small\sffamily, % размер и начертание шрифта для подсветки кода
	numbers=left,               % где поставить нумерацию строк (слева\справа)
	numberstyle=\tiny,           % размер шрифта для номеров строк
	stepnumber=1,                   % размер шага между двумя номерами строк
	numbersep=5pt,                % как далеко отстоят номера строк от подсвечиваемого кода
	showspaces=false,            % показывать или нет пробелы специальными отступами
	showstringspaces=false,      % показывать или нет пробелы в строках
	showtabs=false,             % показывать или нет табуляцию в строках
	frame=single,              % рисовать рамку вокруг кода
	tabsize=2,                 % размер табуляции по умолчанию равен 2 пробелам
	captionpos=t,              % позиция заголовка вверху [t] или внизу [b] 
	breaklines=true,           % автоматически переносить строки (да\нет)
	breakatwhitespace=false, % переносить строки только если есть пробел
	escapeinside={\#*}{*)}   % если нужно добавить комментарии в коде
}


\usepackage[left=3cm,right=1.5cm, top=2cm,bottom=2cm,bindingoffset=0cm]{geometry}
% Для измененных титулов глав:
\usepackage{titlesec, blindtext, color} % подключаем нужные пакеты
\definecolor{gray75}{gray}{0.75} % определяем цвет
\newcommand{\hsp}{\hspace{20pt}} % длина линии в 20pt
% titleformat определяет стиль
\titleformat{\chapter}{\LARGE\bfseries}{\thechapter}{20pt}{\LARGE\bfseries}
\titleformat{\section}{\Large\bfseries}{\thesection}{18pt}{\Large\bfseries}


% plot
\usepackage{pgfplots}
\usepackage{filecontents}
\usetikzlibrary{datavisualization}
\usetikzlibrary{datavisualization.formats.functions}

\begin{document}
	%\def\chaptername{} % убирает "Глава"
	\thispagestyle{empty}
	\begin{titlepage}
		\noindent \begin{minipage}{0.15\textwidth}
			\includegraphics[width=\linewidth]{img/b_logo}
		\end{minipage}
		\noindent\begin{minipage}{0.9\textwidth}\centering
			\textbf{Министерство науки и высшего образования Российской Федерации}\\
			\textbf{Федеральное государственное бюджетное образовательное учреждение высшего образования}\\
			\textbf{~~~«Московский государственный технический университет имени Н.Э.~Баумана}\\
			\textbf{(национальный исследовательский университет)»}\\
			\textbf{(МГТУ им. Н.Э.~Баумана)}
		\end{minipage}
		
		\noindent\rule{18cm}{3pt}
		\newline\newline
		\noindent ФАКУЛЬТЕТ $\underline{\text{«Информатика и системы управления»}}$ \newline\newline
		\noindent КАФЕДРА $\underline{\text{«Программное обеспечение ЭВМ и информационные технологии»}}$\newline\newline\newline\newline\newline
		
		\begin{center}
			\noindent\begin{minipage}{1.1\textwidth}\centering
				\Large\textbf{  Отчет по лабораторной работе №9}\newline
				\textbf{по дисциплине <<Функциональное и логическое}\newline
				\textbf{~~~программирование>>}\newline\newline
			\end{minipage}
		\end{center}
		
		\noindent\textbf{Тема} $\underline{\text{Использование правил в программе на Prolog~~~~~~~~~~~~~}}$\newline\newline
		\noindent\textbf{Студент} $\underline{\text{Завойских Е.В.~~~~~~~~~~~~~~~~~~~~~~~~~~~~~~~~~~~~~~~~~~~~~~~~~~}}$\newline\newline
		\noindent\textbf{Группа} $\underline{\text{ИУ7-63Б~~~~~~~~~~~~~~~~~~~~~~~~~~~~~~~~~~~~~~~~~~~~~~~~~~~~~~~~~~~~}}$\newline\newline
		\noindent\textbf{Оценка (баллы)} $\underline{\text{~~~~~~~~~~~~~~~~~~~~~~~~~~~~~~~~~~~~~~~~~~~~~~~~~~~~~~~~~~~}}$\newline\newline
		\noindent\textbf{Преподаватели} $\underline{\text{Толпинская Н.Б., Строганов Ю.В.~~~~~~~~~~~~~}}$\newline\newline\newline
		
		\begin{center}
			\vfill
			Москва~---~\the\year
			~г.
		\end{center}
	\end{titlepage}
		

\section*{Задание}

\textbf{1. Создать базу знаний «Предки»}, позволяющую наиболее эффективным способом (за меньшее количество шагов, что обеспечивается меньшим количеством предложений БЗ --- правил), и используя разные варианты (примеры) простого вопроса, (указать: какой вопрос для какого варианта) определить:

\begin{enumerate}
    \item по имени субъекта определить всех его бабушек (предки 2-го колена),
    \item по имени субъекта определить всех его дедушек (предки 2-го колена),
    \item по имени субъекта определить всех его бабушек и дедушек (предки 2-го колена),
    \item по имени субъекта определить его бабушку по материнской линии (предки 2-го колена),
    \item по имени субъекта определить его бабушку и дедушку по материнской линии
(предки 2-го колена).
\end{enumerate}

Минимизировать количество правил и количество вариантов вопросов. Использовать конъюнктивные правила и простой вопрос. Для одного из вариантов ВОПРОСА задания 1 составить таблицу, отражающую конкретный порядок работы системы.

\begin{lstlisting}[language=Prolog]
domains 	
	name, sex = string.  
	
predicates 	
	parent(name, name, sex).         
	grandparent(name, name, sex, sex). 
	
clauses 
	parent("child", "mom", "w").
	parent("child", "dad", "m").
	parent("mom", "mom of mom", "w").
	parent("mom", "dad of mom", "m").
	parent("dad", "mom of dad", "w").
	parent("dad", "dad of dad", "m").
	
	grandparent(Child, Grandparent, ParentSex, GrandparentSex) :- parent(Child, Parent, ParentSex), parent(Parent, Grandparent, GrandparentSex).

goal 	
	%grandparent("child", Name, _, "w"). 
	%grandparent("child", Name, _, "m"). 
	%grandparent("child", Name, _, _). 
	%grandparent("child", Name, "w", "w"). 
	grandparent("child", Name, "w", _).  
 
\end{lstlisting}

\textbf{2. Дополнить базу знаний правилами, позволяющими найти}

1. Максимум из двух чисел

а) без использования отсечения,

в) с использованием отсечения;

2. Максимум из трех чисел

а) без использования отсечения,

в) с использованием отсечения;

Убедиться в правильности результатов.

Для каждого случая пункта 2 обосновать необходимость всех условий тела.
Для одного из вариантов ВОПРОСА и каждого варианта задания 2 составить
таблицу, отражающую конкретный порядок работы системы.

Для одного из вариантов ВОПРОСА и конкретной БЗ составить таблицу, отражающую конкретный порядок работы системы, с объяснениями: очередная проблема на каждом шаге и метод ее решения; каково новое текущее состояние резольвенты, как получено; какие дальнейшие действия? (Запускается ли алгоритм унификации? Каких термов? Почему этих?); вывод по результатам очередного шага и дальнейшие действия.

\begin{lstlisting}[language=Prolog]
domains
	num = integer.

predicates
	max_1_1(num, num, num).
	max_1_2(num, num, num).
	max_2_1(num, num, num, num).
	max_2_2(num, num, num, num).
	
clauses
	max_1_1(N1, N2, N2) :- N2 >= N1.
	max_1_1(N1, N2, N1) :- N1 > N2.
	
	max_1_2(N1, N2, N2) :- N2 >= N1, !.
	max_1_2(N1, _, N1).
	
	max_2_1(N1, N2, N3, N1) :- N1 >= N2, N1 >= N3.
	max_2_1(N1, N2, N3, N2) :- N2 >= N1, N2 >= N3.
	max_2_1(N1, N2, N3, N3) :- N3 >= N1, N3 >= N2.
	
	max_2_2(N1, N2, N3, N1) :- N1 >= N2, N1 >= N3, !.
	max_2_2(_, N2, N3, N2) :- N2 >= N3, !.
	max_2_2(_, _, N3, N3).
	
goal
	%max_1_1(1, 3, Max).
	%max_1_1(3, 1, Max).
	
	%max_1_2(1, 3, Max).
	%max_1_2(3, 1, Max).
	
	%max_2_1(1, 2, 3, Max).
	%max_2_1(1, 3, 2, Max).
	%max_2_1(3, 2, 1, Max).
	
	%max_2_2(1, 2, 3, Max).
	%max_2_2(1, 3, 2, Max).
	max_2_2(3, 2, 1, Max).
 
\end{lstlisting}



\bibliographystyle{utf8gost705u}  % стилевой файл для оформления по ГОСТу
\bibliography{51-biblio}          % имя библиографической базы (bib-файла)
	
\end{document}
