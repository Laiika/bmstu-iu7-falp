

\documentclass[12pt]{report}
\usepackage[utf8]{inputenc}
\usepackage[russian]{babel}
\usepackage[14pt]{extsizes}
\usepackage{listings}
\usepackage{graphicx}
\usepackage{amsmath,amsfonts,amssymb,amsthm,mathtools} 
\usepackage{pgfplots}
\usepackage{filecontents}
\usepackage{float}
\usepackage{indentfirst}
\usepackage{eucal}
\usepackage{enumitem}
%s\documentclass[openany]{book}
\frenchspacing

\usepackage{titlesec}
\titleformat{\section}
{\normalsize\bfseries}
{\thesection}
{1em}{}
\titlespacing*{\chapter}{0pt}{-30pt}{8pt}
\titlespacing*{\section}{\parindent}{*4}{*4}
\titlespacing*{\subsection}{\parindent}{*4}{*4}

\usepackage{indentfirst} % Красная строка

\usetikzlibrary{datavisualization}
\usetikzlibrary{datavisualization.formats.functions}

\usepackage{amsmath}

\usepackage{graphicx}
\newcommand{\img}[3] {
    \begin{figure}[h]
        \center{\includegraphics[height=#1]{img/#2}}
        \caption{#3}
        \label{img:#2}
    \end{figure}
}


% Для листинга кода:
\lstset{ %
	language=prolog,                 % выбор языка для подсветки 
	basicstyle=\small\sffamily, % размер и начертание шрифта для подсветки кода
	numbers=left,               % где поставить нумерацию строк (слева\справа)
	numberstyle=\tiny,           % размер шрифта для номеров строк
	stepnumber=1,                   % размер шага между двумя номерами строк
	numbersep=5pt,                % как далеко отстоят номера строк от подсвечиваемого кода
	showspaces=false,            % показывать или нет пробелы специальными отступами
	showstringspaces=false,      % показывать или нет пробелы в строках
	showtabs=false,             % показывать или нет табуляцию в строках
	frame=single,              % рисовать рамку вокруг кода
	tabsize=2,                 % размер табуляции по умолчанию равен 2 пробелам
	captionpos=t,              % позиция заголовка вверху [t] или внизу [b] 
	breaklines=true,           % автоматически переносить строки (да\нет)
	breakatwhitespace=false, % переносить строки только если есть пробел
	escapeinside={\#*}{*)}   % если нужно добавить комментарии в коде
}


\usepackage[left=3cm,right=1.5cm, top=2cm,bottom=2cm,bindingoffset=0cm]{geometry}
% Для измененных титулов глав:
\usepackage{titlesec, blindtext, color} % подключаем нужные пакеты
\definecolor{gray75}{gray}{0.75} % определяем цвет
\newcommand{\hsp}{\hspace{20pt}} % длина линии в 20pt
% titleformat определяет стиль
\titleformat{\chapter}{\LARGE\bfseries}{\thechapter}{20pt}{\LARGE\bfseries}
\titleformat{\section}{\Large\bfseries}{\thesection}{18pt}{\Large\bfseries}


% plot
\usepackage{pgfplots}
\usepackage{filecontents}
\usetikzlibrary{datavisualization}
\usetikzlibrary{datavisualization.formats.functions}

\begin{document}
	%\def\chaptername{} % убирает "Глава"
	\thispagestyle{empty}
	\begin{titlepage}
		\noindent \begin{minipage}{0.15\textwidth}
			\includegraphics[width=\linewidth]{img/b_logo}
		\end{minipage}
		\noindent\begin{minipage}{0.9\textwidth}\centering
			\textbf{Министерство науки и высшего образования Российской Федерации}\\
			\textbf{Федеральное государственное бюджетное образовательное учреждение высшего образования}\\
			\textbf{~~~«Московский государственный технический университет имени Н.Э.~Баумана}\\
			\textbf{(национальный исследовательский университет)»}\\
			\textbf{(МГТУ им. Н.Э.~Баумана)}
		\end{minipage}
		
		\noindent\rule{18cm}{3pt}
		\newline\newline
		\noindent ФАКУЛЬТЕТ $\underline{\text{«Информатика и системы управления»}}$ \newline\newline
		\noindent КАФЕДРА $\underline{\text{«Программное обеспечение ЭВМ и информационные технологии»}}$\newline\newline\newline\newline\newline
		
		\begin{center}
			\noindent\begin{minipage}{1.1\textwidth}\centering
				\Large\textbf{  Отчет по лабораторной работе №11}\newline
				\textbf{по дисциплине <<Функциональное и логическое}\newline
				\textbf{~~~программирование>>}\newline\newline
			\end{minipage}
		\end{center}
		
		\noindent\textbf{Тема} $\underline{\text{Рекурсия на Prolog~~~~~~~~~~~~~~~~~~~~~~~~~~~~~~~~~~~~~~~~~~~~~~~~~}}$\newline\newline
		\noindent\textbf{Студент} $\underline{\text{Завойских Е.В.~~~~~~~~~~~~~~~~~~~~~~~~~~~~~~~~~~~~~~~~~~~~~~~~~~}}$\newline\newline
		\noindent\textbf{Группа} $\underline{\text{ИУ7-63Б~~~~~~~~~~~~~~~~~~~~~~~~~~~~~~~~~~~~~~~~~~~~~~~~~~~~~~~~~~~~}}$\newline\newline
		\noindent\textbf{Оценка (баллы)} $\underline{\text{~~~~~~~~~~~~~~~~~~~~~~~~~~~~~~~~~~~~~~~~~~~~~~~~~~~~~~~~~~~}}$\newline\newline
		\noindent\textbf{Преподаватели} $\underline{\text{Толпинская Н.Б., Строганов Ю.В.~~~~~~~~~~~~~}}$\newline\newline\newline
		
		\begin{center}
			\vfill
			Москва~---~\the\year
			~г.
		\end{center}
	\end{titlepage}
		

\section*{Задание}

Используя хвостовую рекурсию, разработать (комментируя назначение
аргументов) эффективную программу , позволяющую:

1. Найти длину списка (по верхнему уровню);

2. Найти сумму элементов числового списка;

3. Найти сумму элементов числового списка, стоящих на нечетных позициях исходного списка (нумерация от 0);

4. Сформировать список из элементов числового списка, больших заданного значения;

5. Удалить заданный элемент из списка (один или все вхождения);

6. Объединить два списка.

\begin{lstlisting}[language=Prolog]
domains
	list = integer*.
	num = integer.

predicates
	len_r(list, num, num).
	len(list, num).
	
	sum_r(list, num, num).
	sum(list, num).
	
	sum2_r(list, num, num).
	sum2(list, num).
	
	els_large_r(list, list, num, list).
	els_large(list, list, num).
	
	del_el_r(list, list, num, list).
	del_el(list, list, num).
	
	merge(list, list, list).
	
	
clauses
	% 1
	len_r([], Res, Res) :- !.
	len_r([_|T], Res, Cur) :- Cur2 = Cur + 1, 
			   	  len_r(T, Res, Cur2).
	
	len(L, Res) :- len_r(L, Res, 0).
	
	%2
	sum_r([], Res, Res) :- !.
	sum_r([H|T], Res, Cur) :- Cur2 = Cur + H,
				  sum_r(T, Res, Cur2).
				  
	sum(L, Res) :- sum_r(L, Res, 0).
	
	%3
	sum2_r([], Res, Res) :- !.
	
	sum2_r([_], Res, Res) :- !.
	
	sum2_r([_, H|T], Res, Cur) :- Cur2 = Cur + H,
				      sum2_r(T, Res, Cur2).
				  
	sum2(L, Res) :- sum2_r(L, Res, 0).
	
	%4
	els_large_r([], Res, _, Res) :- !.
	
	els_large_r([H|T], Res, N, Cur) :- H > N, !,
				  	   els_large_r(T, Res, N, [H|Cur]).
				  	   
	els_large_r([_|T], Res, N, Cur) :- els_large_r(T, Res, N, Cur).
				  	   
				  
	els_large(L, Res, N) :- els_large_r(L, Res, N, []).
	
	%5
	del_el_r([], Res, _, Res) :- !.
	
	del_el_r([H|T], Res, N, Cur) :- H <> N, !,
				  	del_el_r(T, Res, N, [H|Cur]).
				  	   
	del_el_r([_|T], Res, N, Cur) :- del_el_r(T, Res, N, Cur).
				  	   
				  
	del_el(L, Res, N) :- del_el_r(L, Res, N, []).
	
	%6
	merge([], L2, L2) :- !.
	merge([H|T], L2, [H|T3]):- merge(T, L2, T3).
	
goal
	%len([1, 2, 3, 4, 5], Res).
	%sum([1, 2, 3, 4, 5], Res).
	%sum2([1, 2, 3, 4, 5], Res).
	%els_large([3, 4, 5, 1, 0], Res, 2).
	%del_el([3, 4, 5, 1, 0, 4], Res, 4).
	merge([3, 4, 5], [1, 0], Res).
\end{lstlisting}


\bibliographystyle{utf8gost705u}  % стилевой файл для оформления по ГОСТу
\bibliography{51-biblio}          % имя библиографической базы (bib-файла)
	
\end{document}
