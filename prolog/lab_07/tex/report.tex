

\documentclass[12pt]{report}
\usepackage[utf8]{inputenc}
\usepackage[russian]{babel}
\usepackage[14pt]{extsizes}
\usepackage{listings}
\usepackage{graphicx}
\usepackage{amsmath,amsfonts,amssymb,amsthm,mathtools} 
\usepackage{pgfplots}
\usepackage{filecontents}
\usepackage{float}
\usepackage{indentfirst}
\usepackage{eucal}
\usepackage{enumitem}
%s\documentclass[openany]{book}
\frenchspacing

\usepackage{titlesec}
\titleformat{\section}
{\normalsize\bfseries}
{\thesection}
{1em}{}
\titlespacing*{\chapter}{0pt}{-30pt}{8pt}
\titlespacing*{\section}{\parindent}{*4}{*4}
\titlespacing*{\subsection}{\parindent}{*4}{*4}

\usepackage{indentfirst} % Красная строка

\usetikzlibrary{datavisualization}
\usetikzlibrary{datavisualization.formats.functions}

\usepackage{amsmath}

\usepackage{graphicx}
\newcommand{\img}[3] {
    \begin{figure}[h]
        \center{\includegraphics[height=#1]{img/#2}}
        \caption{#3}
        \label{img:#2}
    \end{figure}
}


% Для листинга кода:
\lstset{ %
	language=prolog,                 % выбор языка для подсветки 
	basicstyle=\small\sffamily, % размер и начертание шрифта для подсветки кода
	numbers=left,               % где поставить нумерацию строк (слева\справа)
	numberstyle=\tiny,           % размер шрифта для номеров строк
	stepnumber=1,                   % размер шага между двумя номерами строк
	numbersep=5pt,                % как далеко отстоят номера строк от подсвечиваемого кода
	showspaces=false,            % показывать или нет пробелы специальными отступами
	showstringspaces=false,      % показывать или нет пробелы в строках
	showtabs=false,             % показывать или нет табуляцию в строках
	frame=single,              % рисовать рамку вокруг кода
	tabsize=2,                 % размер табуляции по умолчанию равен 2 пробелам
	captionpos=t,              % позиция заголовка вверху [t] или внизу [b] 
	breaklines=true,           % автоматически переносить строки (да\нет)
	breakatwhitespace=false, % переносить строки только если есть пробел
	escapeinside={\#*}{*)}   % если нужно добавить комментарии в коде
}


\usepackage[left=3cm,right=1.5cm, top=2cm,bottom=2cm,bindingoffset=0cm]{geometry}
% Для измененных титулов глав:
\usepackage{titlesec, blindtext, color} % подключаем нужные пакеты
\definecolor{gray75}{gray}{0.75} % определяем цвет
\newcommand{\hsp}{\hspace{20pt}} % длина линии в 20pt
% titleformat определяет стиль
\titleformat{\chapter}{\LARGE\bfseries}{\thechapter}{20pt}{\LARGE\bfseries}
\titleformat{\section}{\Large\bfseries}{\thesection}{18pt}{\Large\bfseries}


% plot
\usepackage{pgfplots}
\usepackage{filecontents}
\usetikzlibrary{datavisualization}
\usetikzlibrary{datavisualization.formats.functions}

\begin{document}
	%\def\chaptername{} % убирает "Глава"
	\thispagestyle{empty}
	\begin{titlepage}
		\noindent \begin{minipage}{0.15\textwidth}
			\includegraphics[width=\linewidth]{img/b_logo}
		\end{minipage}
		\noindent\begin{minipage}{0.9\textwidth}\centering
			\textbf{Министерство науки и высшего образования Российской Федерации}\\
			\textbf{Федеральное государственное бюджетное образовательное учреждение высшего образования}\\
			\textbf{~~~«Московский государственный технический университет имени Н.Э.~Баумана}\\
			\textbf{(национальный исследовательский университет)»}\\
			\textbf{(МГТУ им. Н.Э.~Баумана)}
		\end{minipage}
		
		\noindent\rule{18cm}{3pt}
		\newline\newline
		\noindent ФАКУЛЬТЕТ $\underline{\text{«Информатика и системы управления»}}$ \newline\newline
		\noindent КАФЕДРА $\underline{\text{«Программное обеспечение ЭВМ и информационные технологии»}}$\newline\newline\newline\newline\newline
		
		\begin{center}
			\noindent\begin{minipage}{1.1\textwidth}\centering
				\Large\textbf{  Отчет по лабораторной работе №7}\newline
				\textbf{по дисциплине <<Функциональное и логическое}\newline
				\textbf{~~~программирование>>}\newline\newline
			\end{minipage}
		\end{center}
		
		\noindent\textbf{Тема} $\underline{\text{Среда Visual Prolog~~~~~~~~~~~~~~~~~~~~~~~~~~~~~~~~~~~~~~~~~~~~~~~~~}}$\newline\newline
		\noindent\textbf{Студент} $\underline{\text{Завойских Е.В.~~~~~~~~~~~~~~~~~~~~~~~~~~~~~~~~~~~~~~~~~~~~~~~~~~}}$\newline\newline
		\noindent\textbf{Группа} $\underline{\text{ИУ7-63Б~~~~~~~~~~~~~~~~~~~~~~~~~~~~~~~~~~~~~~~~~~~~~~~~~~~~~~~~~~~~}}$\newline\newline
		\noindent\textbf{Оценка (баллы)} $\underline{\text{~~~~~~~~~~~~~~~~~~~~~~~~~~~~~~~~~~~~~~~~~~~~~~~~~~~~~~~~~~~}}$\newline\newline
		\noindent\textbf{Преподаватели} $\underline{\text{Толпинская Н.Б., Строганов Ю.В.~~~~~~~~~~~~~}}$\newline\newline\newline
		
		\begin{center}
			\vfill
			Москва~---~\the\year
			~г.
		\end{center}
	\end{titlepage}
		

\section{Задание}

Разработать свою программу --- «Телефонный справочник». Абоненты могут иметь
несколько телефонов. Протестировать работу программы, используя разные вопросы.

«Телефонный справочник»: Фамилия, №тел, Адрес --- структура (Город, Улица №дома, №кв).

«Автомобили»: Фамилия\_владельца, Марка, Цвет, Стоимость, Номер.

Владелец может иметь несколько телефонов, автомобилей (Факты). В разных городах
есть однофамильцы, в одном городе --- фамилия уникальна.

Используя конъюнктивное правило и простой вопрос, обеспечить возможность
поиска: По Марке и Цвету автомобиля найти Фамилию, Город, Телефон.

\begin{lstlisting}[language=Prolog]
domains
	address = address_struct(string, string, integer, integer)

predicates
	phone_dir(string, string, address)
	cars(string, string, string, integer, integer)
	get_by_brand_and_color(string, string, address, string, string)

clauses
	phone_dir("Kozlov", "+79876576577", address_struct("Moscow", "Fedina", 55, 23)).
	phone_dir("Kozlov", "+79876576000", address_struct("Saint-Petersburg", "Mira", 4, 12)).
	phone_dir("Sabirova", "+79800006577", address_struct("Kazan", "Leninskaya", 31, 33)).
	phone_dir("Sabirova", "+79800003345", address_struct("Kazan", "Leninskaya", 31, 33)).
	phone_dir("Orehov", "+79876589577", address_struct("Saint-Petersburg", "Annikova", 23, 4)).
	phone_dir("Malkov", "+79876576444", address_struct("Nizhny Novgorod", "Annikova", 48, 1)).
	
	cars("Kozlov", "bmw", "black", 10000, 123).
	cars("Kozlov", "mersedes", "yellow", 30000, 545).
	cars("Sabirova", "lada", "red", 3000, 432).
	cars("Orehov", "tesla", "black", 100000, 133).
	cars("Orehov", "audi", "blue", 20000, 435).
	cars("Malkov", "mersedes", "yellow", 1200, 333).
	
	get_by_brand_and_color(Surname, Phone, Address, Brand, Color) :- phone_dir(Surname, Phone, Address), cars(Surname, Brand, Color, _, _).

goal
	%get_by_brand_and_color(Surname, Phone, address_struct(City, _, _, _), "lada", "red").
	get_by_brand_and_color(Surname, Phone, address_struct(City, _, _, _), "mersedes", "yellow").
\end{lstlisting}


\bibliographystyle{utf8gost705u}  % стилевой файл для оформления по ГОСТу
\bibliography{51-biblio}          % имя библиографической базы (bib-файла)
	
\end{document}
