

\documentclass[12pt]{report}
\usepackage[utf8]{inputenc}
\usepackage[russian]{babel}
\usepackage[14pt]{extsizes}
\usepackage{listings}
\usepackage{graphicx}
\usepackage{amsmath,amsfonts,amssymb,amsthm,mathtools} 
\usepackage{pgfplots}
\usepackage{filecontents}
\usepackage{float}
\usepackage{indentfirst}
\usepackage{eucal}
\usepackage{enumitem}
%s\documentclass[openany]{book}
\frenchspacing

\usepackage{titlesec}
\titleformat{\section}
{\normalsize\bfseries}
{\thesection}
{1em}{}
\titlespacing*{\chapter}{0pt}{-30pt}{8pt}
\titlespacing*{\section}{\parindent}{*4}{*4}
\titlespacing*{\subsection}{\parindent}{*4}{*4}

\usepackage{indentfirst} % Красная строка

\usetikzlibrary{datavisualization}
\usetikzlibrary{datavisualization.formats.functions}

\usepackage{amsmath}

\usepackage{graphicx}
\newcommand{\img}[3] {
    \begin{figure}[h]
        \center{\includegraphics[height=#1]{img/#2}}
        \caption{#3}
        \label{img:#2}
    \end{figure}
}


% Для листинга кода:
\lstset{ %
	language=c,                 % выбор языка для подсветки (здесь это С)
	basicstyle=\small\sffamily, % размер и начертание шрифта для подсветки кода
	numbers=left,               % где поставить нумерацию строк (слева\справа)
	numberstyle=\tiny,           % размер шрифта для номеров строк
	stepnumber=1,                   % размер шага между двумя номерами строк
	numbersep=5pt,                % как далеко отстоят номера строк от подсвечиваемого кода
	showspaces=false,            % показывать или нет пробелы специальными отступами
	showstringspaces=false,      % показывать или нет пробелы в строках
	showtabs=false,             % показывать или нет табуляцию в строках
	frame=single,              % рисовать рамку вокруг кода
	tabsize=2,                 % размер табуляции по умолчанию равен 2 пробелам
	captionpos=t,              % позиция заголовка вверху [t] или внизу [b] 
	breaklines=true,           % автоматически переносить строки (да\нет)
	breakatwhitespace=false, % переносить строки только если есть пробел
	escapeinside={\#*}{*)}   % если нужно добавить комментарии в коде
}


\usepackage[left=3cm,right=1.5cm, top=2cm,bottom=2cm,bindingoffset=0cm]{geometry}
% Для измененных титулов глав:
\usepackage{titlesec, blindtext, color} % подключаем нужные пакеты
\definecolor{gray75}{gray}{0.75} % определяем цвет
\newcommand{\hsp}{\hspace{20pt}} % длина линии в 20pt
% titleformat определяет стиль
\titleformat{\chapter}{\LARGE\bfseries}{\thechapter}{20pt}{\LARGE\bfseries}
\titleformat{\section}{\Large\bfseries}{\thesection}{18pt}{\Large\bfseries}


% plot
\usepackage{pgfplots}
\usepackage{filecontents}
\usetikzlibrary{datavisualization}
\usetikzlibrary{datavisualization.formats.functions}

\begin{document}
	%\def\chaptername{} % убирает "Глава"
	\thispagestyle{empty}
	\begin{titlepage}
		\noindent \begin{minipage}{0.15\textwidth}
			\includegraphics[width=\linewidth]{img/b_logo}
		\end{minipage}
		\noindent\begin{minipage}{0.9\textwidth}\centering
			\textbf{Министерство науки и высшего образования Российской Федерации}\\
			\textbf{Федеральное государственное бюджетное образовательное учреждение высшего образования}\\
			\textbf{~~~«Московский государственный технический университет имени Н.Э.~Баумана}\\
			\textbf{(национальный исследовательский университет)»}\\
			\textbf{(МГТУ им. Н.Э.~Баумана)}
		\end{minipage}
		
		\noindent\rule{18cm}{3pt}
		\newline\newline
		\noindent ФАКУЛЬТЕТ $\underline{\text{«Информатика и системы управления»}}$ \newline\newline
		\noindent КАФЕДРА $\underline{\text{«Программное обеспечение ЭВМ и информационные технологии»}}$\newline\newline\newline\newline\newline
		
		\begin{center}
			\noindent\begin{minipage}{1.1\textwidth}\centering
				\Large\textbf{  Отчет по лабораторной работе №4}\newline
				\textbf{по дисциплине <<Функциональное и логическое}\newline
				\textbf{~~~программирование>>}\newline\newline
			\end{minipage}
		\end{center}
		
		\noindent\textbf{Тема} $\underline{\text{Использование управляющих структур, работа со списками}}$\newline\newline
		\noindent\textbf{Студент} $\underline{\text{Завойских Е.В.~~~~~~~~~~~~~~~~~~~~~~~~~~~~~~~~~~~~~~~~~~~~~~~~~~}}$\newline\newline
		\noindent\textbf{Группа} $\underline{\text{ИУ7-63Б~~~~~~~~~~~~~~~~~~~~~~~~~~~~~~~~~~~~~~~~~~~~~~~~~~~~~~~~~~~~}}$\newline\newline
		\noindent\textbf{Оценка (баллы)} $\underline{\text{~~~~~~~~~~~~~~~~~~~~~~~~~~~~~~~~~~~~~~~~~~~~~~~~~~~~~~~~~~~}}$\newline\newline
		\noindent\textbf{Преподаватели} $\underline{\text{Толпинская Н.Б., Строганов Ю.В.~~~~~~~~~~~~~}}$\newline\newline\newline
		
		\begin{center}
			\vfill
			Москва~---~\the\year
			~г.
		\end{center}
	\end{titlepage}
	
\chapter{Теоретические вопросы}
	
\section{Синтаксическая форма и хранение программы в памяти}

Синтаксически программа оформляется в виде S-выражения, которое может быть структурированным. Наличие скобок является признаком структуры. 

По определению:
\begin{itemize}
	\item S-выражение ::= <атом> | <точечная пара>

	\item Атомы:
\begin{itemize} 
	\item символы (идентификаторы) --- синтаксически --- набор литер (букв и цифр), начинающихся с буквы;
	\item специальные символы --- {T, Nil} (используются для обозначения логических констант);
	\item самоопределимые атомы --- натуральные числа, дробные числа, вещественные числа, строки --- последовательность символов, заключенных в двойные апострофы (например, “abc”);
\end{itemize} 


\item Точечная пара ::= (<атом> . <атом>) | (<атом> . <точечная пара>) | (<точечная пара> . <атом>) | (<точечная пара> . <точечная пара>);

\item Список ::= <пустой список> | <непустой список>, где 

<пустой список> ::= () | Nil,

<непустой список> ::= (<первый элемент> . <хвост>),

<первый элемент> ::= <S-выражение>,

<хвост> ::= <список>.

\end{itemize}

Атомы представляются в памяти пятью указателями: name, value, function, property, package; точечная пара --- списковой ячейкой (бинарным узлом), хранящей два указателя: car-указатель (на голову) и cdr-указатель (на хвост).


\section{Трактовка элементов списка}

По умолчанию список является формой (вычислимым выражением), в которой первый
элемент трактуется как имя фунции, остальные элементы --- как ее аргументы.  Чтобы была возможность отличить программу от данных создана функция quote и ее сокращенное обозначение --- апостроф \texttt{'}. Функция quote блокирует
вычисление своего аргумента и возращает его текстовую запись.


\section{Порядок реализации программы}

Лисп-программа включает в себя:
\begin{itemize}
	\item определения новых функций на базе встроенных функций и других функций, определённых в этой программе;
	\item вызовы этих новых функций для конкретных значений их аргументов.
\end{itemize}

Лисп-программа представляет собой вызов функции на верхнем уровне и синтаксически оформляется в виде S-выражения.
Выполнение программы заключается в вычислении значений функций для конкретных
значений аргументов. Вычисление функции происходит с помощью функции eval. Функция eval принимает в качестве аргумента S-выражение и вычисляет его.  Если вычислять значение выражения не нужно, используют
функцию quote.


\section{Способы определения функции}

\begin{itemize}
    \item с использованием $\lambda$-нотации (функции без имени)
	
            $\lambda$-выражение: (lambda $\lambda$-список тело\_функции), 
            где $\lambda$-список --- формальные параметры функции.
	
            Вызов такой функции осуществляется следующим способом: ($\lambda$-выражение фактические параметры).

            Вычисление функций без имени может быть выполнено с использованием функционала apply: (apply $\lambda$-выражение список\_фактических\_параметров); или с использованием функционала funcall: (funcall $\lambda$-выражение фактические\_параметры).
	
    \item с использованием макро-определения defun: 
	
	(defun имя\_функции $\lambda$-выражение), 
	
	или  в облегченной форме:
	
	(defun имя\_функции $(x_1, x_2, ..., x_k)$ тело\_функции), 
	где $(x_1, x_2, ..., x_k)$ --- список аргументов.
	
	В качестве имени функции выступает символьный атом. 
	Вызов именованной функции осуществляется следующим образом: (имя\_функции фактические параметры).
\end{itemize} 


\section{Работа со списками}

В Lisp существует множество функций для работы со списками:

\begin{itemize}
    \item (cons arg1 arg2) --- создает списковую ячейку. Если arg2 --- список, то функция вернет список, иначе --- точечную пару.
    \item (list arg1 arg2 arg3 ...) --- создает список из значений своих аргументов.
    \item (append lst1 lst2 lst3 ...) --- объединяет списки (работает с копиями
списковых ячеек, не копируется только последний список-аргумент). nconc ---альтернатива (конкатенация самих списков, а не их копий).
    \item (reverse lst) --- переворачивает список в обратную сторону по верхнему уровню. nreverse --- альтернатива (работает с самим списком, не создает копию).
    \item (last lst) --- возвращает последнюю списковую ячейку.
    \item (nth n lst) --- возвращает n-ый элемент списка.
    \item (nthcdr n lst) --- возвращает n-ый хвост списка.
    \item (length lst) --- возвращает длину списка по верхнему уровню.
    \item (remove elem lst) --- удаляет списковую ячейку, используя функцию
eql.
    \item (rplaca lst elem) --- меняет первый элемент списка на elem.
    \item (rplacd lst arg) --- меняет хвост списка на arg.
    \item (member elem lst) --- возвращает список, начиная с найденного элемента
elem (или Nil).
    \item (union lst1 lst2) --- объединение множеств.
    \item (intersection lst1 lst2) --- пересечение множеств.
    \item (set-difference lst1 lst2) --- дополнение lst1 до lst2.
    \item (assoc elem table) --- возвращает точечную пару, в которой ключ равен
elem.
    \item (rassoc elem table) --- возвращает точечную пару, в которой значение
равно elem.
\end{itemize}

	
\chapter{Практические задания}	

\section{Чем принципиально отличаются функции cons, list, append?
Пусть (setf lst1 '( a b c))
(setf lst2 '( d e)).
Каковы результаты вычисления следующих выражений?
 (cons lst1 lst2)
 (list lst1 lst2)
 (append lst1 lst2)}
 
cons принимает 2 аргумента, создает списковую ячейку и ставит указатели
на 2 аргумента, таким образом объединяя их в точечную пару.

list, составляющая список из значений своих аргументов, создает столько списковых ячеек, сколько аргументов ей было передано. Эта функция относится к особым, поскольку у неё может быть произвольное число аргументов.

append принимает произвольное количество аргументов-списков и соединяет элементы верхнего уровня всех списков в один список. Действие append иногда называют конкатенацией списков. Функцией создаются копии всех списков, кроме последнего. В результате должен быть построен новый список.
Если последний переданный список будет модифицирован, то  итоговый список будет также изменен.

Основные отличия: list и append принимают произвольное количество аргументов, cons --- фиксированное (два); cons создает одну списковую ячейку (в зависимости от второго аргумента может получится список или точечная пара), list и append --- список; cons и list создают новые списковые ячейки (все), а append имеет общие списковые ячейки с последним списком.


\begin{lstlisting}[language=Lisp]
(setf lst1 '( a b c))
(setf lst2 '( d e))

(cons lst1 lst2) -> ((a b c) d e)
(list lst1 lst2) -> ((a b c) (d e))
(append lst1 lst2) -> (a b c d e)

\end{lstlisting}


\section{Каковы результаты вычисления следующих выражений, и почему?}

\textbf{reverse} переворачивает свой список-аргумент, т.е. меняет порядок его элементов  верхнего уровня на противоположный.

\textbf{last} возвращает последнюю cons-ячейку в списке. Если вызывается с целочисленным аргументом n, возвращает n ячеек (по умолчанию n = 1).
Если n больше или равно количеству cons-ячеек в списке, результатом будет исходный список.

\begin{lstlisting}[language=Lisp]
(reverse '(a b c)) -> (c b a)               
(reverse ()) -> ()
(reverse '(a b (c (d)))) -> ((c (d)) b a)   
(reverse '((a b c))) -> ((a b c))
(reverse '(a)) -> (a)
(last '(a b c)) -> (c)                      
(last '(a b (c))) -> ((c))
(last '(a)) -> (a)                          
(last ()) -> ()
(last '((a b c))) -> ((a b c))

\end{lstlisting}
 

\section{Написать, по крайней мере, два варианта функции, которая возвращает
последний элемент своего списка-аргумента}

\begin{lstlisting}[language=Lisp]
(defun f1 (lst) 
    (car (last lst)) )

(defun f2 (lst) 
    (car (reverse lst)) )
\end{lstlisting}


\section{Написать, по крайней мере, два варианта функции, которая возвращает
свой список-аргумент без последнего элемента. } 

\begin{lstlisting}[language=Lisp]
(defun f1 (lst) 
    (reverse (cdr (reverse lst))) )

(defun f2 (lst) 
    (if (cdr lst)
        (cons (car lst) (f2 (cdr lst)) ) ) )

(defun f3 (lst) (butlast lst))
\end{lstlisting}


\section{Напишите функцию swap-first-last, которая переставляет в списке-аргументе первый и последний элементы.}

\begin{lstlisting}[language=Lisp]
(defun without-last (lst) 
    (reverse (cdr (reverse lst))) )
    
(defun swap-first-last (lst) 
    (append (last lst) (without-last (cdr lst)) (list (car lst)) ) )
\end{lstlisting}


\section{Написать простой вариант игры в кости, в котором бросаются две
правильные кости. Если сумма выпавших очков равна 7 или 11 ---
выигрыш, если выпало (1,1) или (6,6) --- игрок имеет право снова
бросить кости, во всех остальных случаях ход переходит ко второму
игроку, но запоминается сумма выпавших очков. Если второй игрок не
выигрывает абсолютно, то выигрывает тот игрок, у которого больше
очков. Результат игры и значения выпавших костей выводить на экран с
помощью функции print. }

\begin{lstlisting}[language=Lisp]

(defun get_points () (+ (random 6) 1))

(defun abs_win (sum) (or (= sum 7) (= sum 11)) )

(defun roll_of_dice (player) 
    (let ((ps1 (get_points)) (ps2 (get_points)))
          (print (list 'Player player 'got ps1 ps2))
          (cond 
            ((abs_win (+ ps1 ps2)) 0)
            ((and (= ps1 ps2) (or (= ps1 1) (= ps1 6))) (roll_of_dice player))
            (T (+ ps1 ps2)) ) ) )
  
(defun game_result (res1 res2) 
    (if (= 0 res2)
        (print "Player 2 won absolutely")
        (cond 
            ((< res1 res2) (print "Player 2 won"))
            ((> res1 res2) (print "Player 1 won"))
            (T (print "Draw")) ) ) )

(defun play () 
    (let ((res1 (roll_of_dice 1)))
         (if (= 0 res1) 
             (print "Player 1 won absolutely")
             (game_result res1 (roll_of_dice 2)) ) ) )

\end{lstlisting}


\section{Написать функцию, которая по своему списку-аргументу lst определяет
является ли он палиндромом (то есть равны ли lst и (reverse lst)).}

\begin{lstlisting}[language=Lisp]
; v1
(defun f1 (lst)
    (equal lst (reverse lst)) )

; v2
(defun without-last (lst) 
    (reverse (cdr (reverse lst))) )
    
(defun f2 (lst)
    (if (> (length lst) 1) 
        (and (eql (car lst) (car (reverse lst))) 
             (f2 (cdr (without-last lst))) )
        (and T) ) )

\end{lstlisting}


\section{Напишите свои необходимые функции, которые обрабатывают таблицу из
4-х точечных пар: (страна . столица), и возвращают по стране --- столицу, а по столице --- страну.}

\begin{lstlisting}[language=Lisp]
; 
; v1
(defun get-capital (table country)
    (cdr (assoc country table)) )
;v2
(defun get-capital (table country)
    (cond 
        ((eql (caar table) country) (cdar table))
        ((eql (caadr table) country) (cdadr table)) 
        ((eql (caaddr table) country) (cdaddr table))
        ((eql (caadddr table) country) (cdadddr table)) ) )

;
; v1
(defun get-country (table capital)
    (car (rassoc capital table)) )
; v2
(defun get-country (table capital)
    (cond 
        ((eql (cdar table) capital) (caar table))
        ((eql (cdadr table) capital) (caadr table)) 
        ((eql (cdaddr table) capital) (caaddr table))
        ((eql (cdadddr table) capital) (caadddr table)) ) )
\end{lstlisting}


\section{Напишите функцию, которая умножает на заданное число-аргумент
первый числовой элемент списка из заданного 3-х элементного списка-аргумента, когда a) все элементы списка --- числа, б) элементы списка -- любые объекты.}

\begin{lstlisting}[language=Lisp]
; a
(defun f1 (lst num)
    (cons (* num (car lst)) (cdr lst) ) )

; b
(defun f2 (lst num)
    (cond 
        ((numberp (car lst))  (cons (* num (car lst)) (cdr lst)) )
        ((numberp (cadr lst)) (list (car lst) (* num (cadr lst)) (caddr lst)) )
        ((numberp (caddr lst)) (list (car lst) (cadr lst) (* num (caddr lst))) )
        (T lst) ) )
\end{lstlisting}



\bibliographystyle{utf8gost705u}  % стилевой файл для оформления по ГОСТу
\bibliography{51-biblio}          % имя библиографической базы (bib-файла)
	
\end{document}
