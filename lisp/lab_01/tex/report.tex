

\documentclass[12pt]{report}
\usepackage[utf8]{inputenc}
\usepackage[russian]{babel}
\usepackage[14pt]{extsizes}
\usepackage{listings}
\usepackage{graphicx}
\usepackage{amsmath,amsfonts,amssymb,amsthm,mathtools} 
\usepackage{pgfplots}
\usepackage{filecontents}
\usepackage{float}
\usepackage{indentfirst}
\usepackage{eucal}
\usepackage{enumitem}
%s\documentclass[openany]{book}
\frenchspacing

\usepackage{titlesec}
\titleformat{\section}
{\normalsize\bfseries}
{\thesection}
{1em}{}
\titlespacing*{\chapter}{0pt}{-30pt}{8pt}
\titlespacing*{\section}{\parindent}{*4}{*4}
\titlespacing*{\subsection}{\parindent}{*4}{*4}

\usepackage{indentfirst} % Красная строка

\usetikzlibrary{datavisualization}
\usetikzlibrary{datavisualization.formats.functions}

\usepackage{amsmath}

\usepackage{graphicx}
\newcommand{\img}[3] {
    \begin{figure}[h]
        \center{\includegraphics[height=#1]{img/#2}}
        \caption{#3}
        \label{img:#2}
    \end{figure}
}


% Для листинга кода:
\lstset{ %
	language=c,                 % выбор языка для подсветки (здесь это С)
	basicstyle=\small\sffamily, % размер и начертание шрифта для подсветки кода
	numbers=left,               % где поставить нумерацию строк (слева\справа)
	numberstyle=\tiny,           % размер шрифта для номеров строк
	stepnumber=1,                   % размер шага между двумя номерами строк
	numbersep=5pt,                % как далеко отстоят номера строк от подсвечиваемого кода
	showspaces=false,            % показывать или нет пробелы специальными отступами
	showstringspaces=false,      % показывать или нет пробелы в строках
	showtabs=false,             % показывать или нет табуляцию в строках
	frame=single,              % рисовать рамку вокруг кода
	tabsize=2,                 % размер табуляции по умолчанию равен 2 пробелам
	captionpos=t,              % позиция заголовка вверху [t] или внизу [b] 
	breaklines=true,           % автоматически переносить строки (да\нет)
	breakatwhitespace=false, % переносить строки только если есть пробел
	escapeinside={\#*}{*)}   % если нужно добавить комментарии в коде
}


\usepackage[left=3cm,right=1.5cm, top=2cm,bottom=2cm,bindingoffset=0cm]{geometry}
% Для измененных титулов глав:
\usepackage{titlesec, blindtext, color} % подключаем нужные пакеты
\definecolor{gray75}{gray}{0.75} % определяем цвет
\newcommand{\hsp}{\hspace{20pt}} % длина линии в 20pt
% titleformat определяет стиль
\titleformat{\chapter}{\LARGE\bfseries}{\thechapter}{20pt}{\LARGE\bfseries}
\titleformat{\section}{\Large\bfseries}{\thesection}{18pt}{\Large\bfseries}


% plot
\usepackage{pgfplots}
\usepackage{filecontents}
\usetikzlibrary{datavisualization}
\usetikzlibrary{datavisualization.formats.functions}

\begin{document}
	%\def\chaptername{} % убирает "Глава"
	\thispagestyle{empty}
	\begin{titlepage}
		\noindent \begin{minipage}{0.15\textwidth}
			\includegraphics[width=\linewidth]{img/b_logo}
		\end{minipage}
		\noindent\begin{minipage}{0.9\textwidth}\centering
			\textbf{Министерство науки и высшего образования Российской Федерации}\\
			\textbf{Федеральное государственное бюджетное образовательное учреждение высшего образования}\\
			\textbf{~~~«Московский государственный технический университет имени Н.Э.~Баумана}\\
			\textbf{(национальный исследовательский университет)»}\\
			\textbf{(МГТУ им. Н.Э.~Баумана)}
		\end{minipage}
		
		\noindent\rule{18cm}{3pt}
		\newline\newline
		\noindent ФАКУЛЬТЕТ $\underline{\text{«Информатика и системы управления»}}$ \newline\newline
		\noindent КАФЕДРА $\underline{\text{«Программное обеспечение ЭВМ и информационные технологии»}}$\newline\newline\newline\newline\newline
		
		\begin{center}
			\noindent\begin{minipage}{1.1\textwidth}\centering
				\Large\textbf{  Отчет по лабораторной работе №1}\newline
				\textbf{по дисциплине <<Функциональное и логическое}\newline
				\textbf{~~~программирование>>}\newline\newline
			\end{minipage}
		\end{center}
		
		\noindent\textbf{Тема} $\underline{\text{Списки в Lispe. Использование стандартных функций.}}$\newline\newline
		\noindent\textbf{Студент} $\underline{\text{Завойских Е.В.~~~~~~~~~~~~~~~~~~~~~~~~~~~~~~~~~~~~~~~~~~~~~~~~~~}}$\newline\newline
		\noindent\textbf{Группа} $\underline{\text{ИУ7-63Б~~~~~~~~~~~~~~~~~~~~~~~~~~~~~~~~~~~~~~~~~~~~~~~~~~~~~~~~~~~~}}$\newline\newline
		\noindent\textbf{Оценка (баллы)} $\underline{\text{~~~~~~~~~~~~~~~~~~~~~~~~~~~~~~~~~~~~~~~~~~~~~~~~~~~~~~~~~~~}}$\newline\newline
		\noindent\textbf{Преподаватели} $\underline{\text{Толпинская Н.Б., Строганов Ю.В.~~~~~~~~~~~~~}}$\newline\newline\newline
		
		\begin{center}
			\vfill
			Москва~---~\the\year
			~г.
		\end{center}
	\end{titlepage}
	
\chapter{Теоретические вопросы}
	
\section{Элементы языка: определение, синтаксис, представление в памяти.}
%Элементами языка Lisp являются атомы, списки и точечные пары.
Вся информация (данные и программы) в Lisp представляется в виде символьных выражений --- S-выражений. По определению

S-выражение ::= <атом>|<точечная пара>.

\textbf{Атомы}
\begin{itemize}
    \item символы (идентификаторы) --- синтаксически --- набор литер (букв и цифр), начинающихся с буквы (например, ПримерАтома);
    \item специальные символы --- {T, Nil} (используются для обозначения логических констант);
    \item самоопределимые атомы --- натуральные числа, дробные числа, вещественные числа, строки --- последовательность символов, заключенных в двойные апострофы (например, “abc”);
\end{itemize} 
		
\textbf{Списки и точечные пары (структуры)}

Определения:

Точечная пара ::= (<атом> . <атом>) | (<атом> . <точечная пара>) | 

(<точечная пара> . <атом>) | (<точечная пара> . <точечная пара>);
		
Список ::= <пустой список> | <непустой список>, где 
		
<пустой список> ::= () | Nil,
		
<непустой список> ::= (<первый элемент> . <хвост>),
		
<первый элемент> ::= <S-выражение>,
		
<хвост> ::= <список>.


Синтаксически:

Любая структура (точечная пара или список) заключается в круглые скобки: (A . B) --- точечная пара, (A) --- список из одного элемента. Пустой список изображается как Nil или (); непустой список по определению может быть изображен: (A . (B . (C . (D . ())))),  допустимо изображение списка последовательностью атомов, разделенных пробелами --- (A B C D).

Элементы списка могут быть списками (любой список заключается в круглые скобки), например --- (A (B C) (D С)). Таким образом, синтаксически наличие скобок является признаком структуры --- списка или точечной пары.

Любая непустая структура Lisp в памяти представляется списковой ячейкой, хранящей два указателя: на голову (первый элемент) и хвост --- всё остальное.

\img{25mm}{list1}{Представление в памяти (A B C D)}


\section{Особенности языка Lisp. Структура программы. Символ апостроф.}

Вся информация (данные и программы) в Lisp представляется в виде символьных выражений --- S-выражений.  

Lisp --- язык символьной обработки. В Lisp программа и данные представлены списками. По умолчанию первый элемент списка трактуется как имя функции, а остальные --- как ее аргументы.

Так как и программа, и данные представлены списками, то их нужно различать. Для этого была создана функция quote.

Функция quote блокирует вычисление своего аргумента. Апостроф --- сокращённое обозначение функции quote. Перед самовычислимыми атомами (числами и атомами T, Nil) можно не ставить апостроф.


\section{Базис языка Lisp. Ядро языка.}
Базис --- это минимальный набор конструкций языка, на основе которых могут быть построены остальные.

Базис Lisp:

\begin{itemize}
	\item атомы;
        \item структуры;
	\item базовые функции и базовые функционалы (cons, car, cdr, eval, quote, atom, eq).

\end{itemize}

\textbf{Атомы}
\begin{itemize}
    \item символы (идентификаторы) --- синтаксически --- набор литер (букв и цифр), начинающихся с буквы (например, ПримерАтома);
    \item специальные символы --- {T, Nil} (используются для обозначения логических констант);
    \item самоопределимые атомы --- натуральные числа, дробные числа, вещественные числа, строки --- последовательность символов, заключенных в двойные апострофы (например, “abc”);
\end{itemize} 

Более сложные данные --- списки и точечные пары (структуры) строятся из унифицированных структур --- блоков памяти --- бинарных узлов.

	
\chapter{Практические задания}	

\section{Представить следующие списки в виде списочных ячеек}

1. '(open close halph)

\img{25mm}{1.1}{Представление в памяти '(open close halph)}

2. '((open1) (close2) (halph3))

\img{35mm}{1.2}{Представление в памяти '((open1) (close2) (halph3))}

3. '((one) for all (and (me (for you))))

\img{43mm}{1.3}{Представление в памяти '((one) for all (and (me (for you))))}

4. '((TOOL) (call))

\clearpage
\img{40mm}{1.4}{Представление в памяти '((TOOL) (call))}

5. '((TOOL1) ((call2)) ((sell)))

\img{45mm}{1.5}{Представление в памяти '((TOOL1) ((call2)) ((sell)))}

6. '(((TOOL) (call)) ((sell)))

\img{45mm}{1.6}{Представление в памяти '(((TOOL) (call)) ((sell)))}

\clearpage
\section{Используя только функции CAR и CDR, написать выражения, возвращающие:}

\begin{enumerate}
        \item Второй элемент заданного списка

        \begin{lstlisting}[language=Lisp]
(CAR (CDR '(A B C D E))) -> B
        \end{lstlisting}

    \item Третий элемент заданного списка

        \begin{lstlisting}[language=Lisp]
(CAR (CDR (CDR '(A B C D E)))) -> C
        \end{lstlisting}

    \item Четвертый элемент заданного списка

        \begin{lstlisting}[language=Lisp]
(CAR (CDR (CDR (CDR '(A B C D E))))) -> D
        \end{lstlisting}
\end{enumerate}
 

\section{Что будет в результате вычисления выражений?}

\begin{enumerate}
    \item (CAADR '((blue cube) (red pyramid))) -> red

        \begin{lstlisting}[language=Lisp]
(CDR '((blue cube) (red pyramid))) -> ((red pyramid))
(CAR '((red pyramid))) -> (red pyramid)
(CAR '(red pyramid)) -> red
        \end{lstlisting}

    \item (CDAR '((abc) (def) (ghi)) -> Nil

        \begin{lstlisting}[language=Lisp]
(CAR '((abc) (def) (ghi))) -> (abc)
(CDR '(abc)) -> Nil
        \end{lstlisting}

    \item (CADR '((abc) (def) (ghi))) -> (def)

        \begin{lstlisting}[language=Lisp]
(CDR '((abc) (def) (ghi))) -> ((def) (ghi))
(CAR '((def) (ghi))) -> (def)
        \end{lstlisting}

    \item (CADDR '((abc) (def) (ghi))) -> (ghi)

        \begin{lstlisting}[language=Lisp]
(CDR '((abc) (def) (ghi))) -> ((def) (ghi))
(CDR '((def) (ghi))) -> ((ghi))
(CAR '((ghi))) -> (ghi)
        \end{lstlisting}
\end{enumerate}


\section{Напишите результат вычисления выражений и объясните, как он получен}

quote блокирует вычисление своего аргумента. Перед  числами и атомами T, Nil можно не ставить апостроф. 

list принимает произвольное число аргументов. Функция создает и возвращает список, состоящий из значений аргументов.

cons имеет фиксированное количество аргументов (два). Если аргументами являются атомы, то создает точечную пару. Если первый аргумент --- атом, а второй --- список, атом становится головой, а второй аргумент (список) становится хвостом. 

\begin{lstlisting}[language=Lisp]
(list 'Fred 'and 'Wilma) -> (Fred and Wilma)
(list 'Fred '(and Wilma)) -> (Fred (and Wilma))
(cons Nil Nil) -> (Nil. Nil) -> (Nil)
(cons T Nil) -> (T. Nil) -> (T)
(cons Nil T) -> (Nil. T)
(list Nil) -> (Nil)
(cons '(T) Nil)	-> ((T). Nil) -> ((T))
(list '(one two) '(free temp)) -> ((one two) (free temp))
	
(cons 'Fred '(and Willma)) -> (Fred and Willma)
(cons 'Fred '(Wilma)) -> (Fred Willma)
(list Nil Nil) -> (Nil Nil)
(list T Nil) -> (T Nil)
(list Nil T) -> (Nil T)
(cons T (list Nil)) -> (T Nil)
(list '(T) Nil) -> ((T) Nil)
(cons '(one two) '(free temp)) -> ((one two) free temp)
\end{lstlisting}

\section{Написать лямбда-выражение и соответствующую функцию. Представить результаты в виде списочных ячеек.}

\begin{enumerate}
    \item Написать функцию (f ar1 ar2 ar3 ar4), возвращающую список: ((ar1 ar2) (ar3 ar4)).
\begin{lstlisting}[language=Lisp] 
(defun f (ar1 ar2 ar3 ar4) (list (list ar1 ar2) (list ar3 ar4)))
(lambda (ar1 ar2 ar3 ar4) (list (list ar1 ar2) (list ar3 ar4)))
\end{lstlisting}

\img{32mm}{1}{Результаты первой функции}

    \item Написать функцию (f ar1 ar2), возвращающую ((ar1) (ar2)).
\begin{lstlisting}[language=Lisp]
(defun f (ar1 ar2) (list (list ar1) (list ar2)))
(lambda (ar1 ar2) (list (list ar1) (list ar2)))
\end{lstlisting}

\img{35mm}{2}{Результаты второй функции}

    \item Написать функцию (f ar1), возвращающую (((ar1))).
\begin{lstlisting}[language=Lisp]
(defun f (ar1) (list (list (list ar1))))
(lambda (ar1) (list (list (list ar1))))
\end{lstlisting}

\img{42mm}{3}{Результаты третьей функции}
\end{enumerate}


\bibliographystyle{utf8gost705u}  % стилевой файл для оформления по ГОСТу
\bibliography{51-biblio}          % имя библиографической базы (bib-файла)
	
\end{document}
